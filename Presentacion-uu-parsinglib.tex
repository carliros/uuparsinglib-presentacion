\documentclass[12pt]{beamer}

\usetheme{Oxygen}
\usepackage{thumbpdf}
\usepackage{wasysym}
\usepackage{ucs}
\usepackage[utf8]{inputenc}
\usepackage{pgf,pgfarrows,pgfnodes,pgfautomata,pgfheaps,pgfshade}
%\usepackage{verbatim}
\usepackage[spanish]{babel}
\usepackage{listings}

\lstdefinelanguage{uuagc}                   % name language
                  []{haskell}               % base language
                  {morekeywords={ATTR       % language keywords            
                                ,SEM
                                ,DATA
                                ,TYPE
                                ,lhs
                                ,loc
                                }
                  ,sensitive=true           % syntax sentitive
                  ,morestring=[d]"          % string delimiters
                  ,morestring=[d]'
                  }

\lstnewenvironment{ag}{\lstset{ language=uuagc
                              , identifierstyle=\ttfamily
                              , showstringspaces=false
                              , basicstyle=\scriptsize
							  , keepspaces=true
                              , stringstyle=\scriptsize
                              , extendedchars=true
                              , commentstyle=\color{blue}
                      }
                      }{}
\lstnewenvironment{hs}{\lstset{ language=haskell
                              , identifierstyle=\ttfamily
                              , showstringspaces=false
							  , basicstyle=\scriptsize
							  , keepspaces=true
                              , stringstyle=\scriptsize
                              , extendedchars=true
                              , commentstyle=\color{blue}
                      }
                  }{}

\setbeamertemplate{navigation symbols}{\insertslidenavigationsymbol}



\pdfinfo
{
  /Title       (uu-parsinglib)
  /Creator     (TeX)
  /Author      (Carlos Gomez)
}


\title[Libreria uu-parsing-lib]{
\textbf{\large Libreria uu-parsinglib}
%\textbf{\large CON}\\
%\textbf{\large HASKELL}
}

%\subtitle{subtitulo}
\author{Carlos Gomez}

\institute[UMSS]{
	Universidad Mayor de San Simón\\
	Facultad de Ciencias y Tecnología\\
	Carrera de Licenciatura en Informática\\
    Comunidad Haskell San Simon}

\date{17 de Septiembre, 2011}

\begin{document}

\frame{\titlepage}

\section*{}
\begin{frame}
  \frametitle{Contenido}
  \tableofcontents[section=1,hidesubsections]
\end{frame}

%\AtBeginSection[]
%{
%  \frame<handout:0>
%  {
%    \frametitle{Contenido}
%    \tableofcontents[currentsection,hideallsubsections]
%  }
%}

%\AtBeginSubsection[]
%{
%  \frame<handout:0>
%  {
%    \frametitle{Contenido}
%    \tableofcontents[sectionstyle=show/hide,subsectionstyle=show/shaded/hide]
%  }
%}

\newcommand<>{\highlighton}[1]{%
  \alt#2{\structure{#1}}{{#1}}
}

\newcommand{\icon}[1]{\pgfimage[height=1em]{#1}}



%%%%%%%%%%%%%%%%%%%%%%%%%%%%%%%%%%%%%%%%%
%%%%%%%%%% Content starts here %%%%%%%%%%
%%%%%%%%%%%%%%%%%%%%%%%%%%%%%%%%%%%%%%%%%

\section{Introducción}

\begin{frame}
\begin{center}
	\textbf{\Large Introducción}
\end{center}
\end{frame}

\subsection{Libreria uu-parsinglib}

\begin{frame}
\frametitle{Libreria uu-parsinglib}

\begin{block}{}
    Es una herramienta \textbf{EDSL\footnote{EDSL: Embeded Domain Specific Language}} para \highlighton{Haskell}
    que permite procesar una entrada a través de una descripción similar a la Gramática Contreta del lenguaje que se 
    quiere procesar.
\end{block}
\end{frame}

\begin{frame}
\frametitle{Beneficios}
	\begin{itemize}
        \item Usar el mismo mecanismo de abstracción, tipado y nombrado de haskell.
        \item Crear |parsers| al vuelo o en tiempo de ejecución del programa.
        \item No depender de otros programas separados para generar el parser. Hacer todo en haskell.
        \item Usar el mismo formalismo para describir scanners y parsers.
	\end{itemize}
\end{frame}

\begin{frame}
\frametitle{Beneficios (Continuacion)}
	\begin{itemize}
        \item Usar el mismo formalismo para describir funciones semánticas y parsers.
        \item Trabajar con versiones limitadas de gramáticas infinitas.
        \item Error-correcting parser combinators.
        \item Disponer de combinadores para trabajar con gramaticas con permutaciones.
        \item Disponer de un interface monadica para conocer informacion intermedia de parsing.
	\end{itemize}
\end{frame}

%\begin{frame}
%\frametitle{Haskell}
%\begin{center}
%	\scalebox{0.4}{\includegraphics{graphics/HaskellLogo.png}}
%\end{center}

\section{Combinadores}

\begin{frame}
\begin{center}
	\textbf{\Large Combinadores}
\end{center}
\end{frame}

\begin{frame}
\frametitle{Combinadores Basicos}
\framesubtitle{Sus definiciones casi no dependen de otros combinadores}
    \begin{itemize}
        \item pSym
        \item pReturn
        \item pFail
        \item Opcional $<<|>$
        \item pAny
	\end{itemize}
\end{frame}

\begin{frame}
\frametitle{Combinadores Derivados}
\framesubtitle{Simples}
Ejemplo de algunos Combinadores Derivados Simples:
    \begin{itemize}
        \item $<\$>$
        \item $<\$$
        \item $<*$
        \item $*>$
	\end{itemize}
\end{frame}

\begin{frame}
\frametitle{Combinadore Derivados}
\framesubtitle{Secuenciales}
Ejemplo de algunos Combinadores Derivados Secuenciales:
    \begin{itemize}
        \item pList, pList\_ng, pListSep, pListSep\_ng
        \item pList1, pList1\_ng, pList1Sep, pList1Sep\_ng
        \item pSome
        \item pMany
        \item pFoldr, pFoldr\_ng, pFoldrSep
        \item pFoldr1, pFoldr1\_ng, pFoldr1Sep
	\end{itemize}
\end{frame}

\begin{frame}
\frametitle{Combinadores Derivados}
\framesubtitle{Especiales}
Ejemplo de algunos Combinadores Derivados Especiales:
    \begin{itemize}
         \item pMaybe
         \item pEither
         \item pPacked
         \item pChainr, pChainl
         \item pCount
	\end{itemize}
\end{frame}

\begin{frame}
\frametitle{Combinadores Derivados}
\framesubtitle{Repetidores}
Ejemplo de algunos Combinadores Derivados Repetidores:
    \begin{itemize}
         \item pExact
         \item pBetween
         \item pAtLeast
         \item pAtMost
	\end{itemize}
\end{frame}

\begin{frame}
\frametitle{Combinadores Derivados}
\framesubtitle{Permutadores}
Ejemplo de algunos Combinadores Derivados Permutadores:
    \begin{itemize}
         \item $<||>$
         \item $<<||>$
         \item pmMany
	\end{itemize}
\end{frame}



\section[Utilitarios]{Utilitarios para construir un Parser}
\begin{frame}
\begin{center}
	\textbf{\Large Utilitarios para construir un parser}
\end{center}
\end{frame}

\subsection[Caracteres]{Parser para caracteres}
\begin{frame}
\frametitle{Utilitarios}
\framesubtitle{Parser para Caracteres Especiales}
Ejemplo de algunos Combinadores especificos para caracteres:
    \begin{itemize}
        \item pLetter, pDigit, pLower, ... , pAscii
        \item pSpaces
	\end{itemize}
\end{frame}

\subsection[Lexeme]{Lexeme Parsers}
\begin{frame}
\frametitle{Lexeme Parsers}
\framesubtitle{Parsers con una ideologia de lexeme}
Ejemplo de algunos Combinadores definidos con la ideologia lexeme:
    \begin{itemize}
        \item pDot
        \item pComma
        \item pLBrace
        \item pSymbol
        \item pInteger
        \item pParens
        \item pTuple
	\end{itemize}
\end{frame}

\subsection[Func. principales]{Funciones Principales para ejecutar el Parser}
\begin{frame}
\frametitle{Funciones Principales de uu-parsinglib}
\framesubtitle{Funciones principales para ejecutar el parser}
Las funciones principales para ejecutar el parser:
    \begin{itemize}
        \item execParser
        \item runParser
	\end{itemize}
\end{frame}

\section[Aplicacion]{Ejemplo de Aplicacion}
\begin{frame}
\begin{center}
	\textbf{\Large Ejemplo de Aplicacion}
\end{center}
\end{frame}

\begin{frame}
\frametitle{Ejemplo de Aplicacion}
Escribir un parser para reconocer un lenguaje de marcado de tipo XML.
\end{frame}

\begin{frame}[fragile]
\frametitle{Ejemplo de Aplicacion}
\framesubtitle{Entrada}
Ejemplo de una entrada para la aplicacion:
\begin{block}{}
\begin{verbatim}
<html>
    <big bold1="ok1" bold2="ok2" >
         image
    </big>
</html>
\end{verbatim}
\end{block}

\end{frame}

\begin{frame}[fragile]
\frametitle{Ejemplo de Aplicacion}
\framesubtitle{Salida}
Ejemplo de una salida para la aplicacion:
\begin{block}{}
\begin{verbatim}
---NTag html
   |
   +---NText "\n    "
   |
   +---NTag big
   |   | bold1 = "ok1"
   |   | bold2 = "ok2"
   |   |
   |   +---NText "\n        image\n    "
   |
   +---NText "\n"
\end{verbatim}
\end{block}
\end{frame}


\begin{frame}[fragile]
\frametitle{Tipo de dato: NTree}
\framesubtitle{Estructura Rosadelfa (NTree)}
\begin{block}{}
\begin{ag}
DATA NTree
  | NTree Node ntrees: NTrees

TYPE NTrees = [NTree]

DATA Node
  | NTag   tag          : String
           attributes   : {[(String,String)]}
  | NText  text         : String
\end{ag}
\end{block}
\end{frame}

\begin{frame}[fragile]
\frametitle{Definicion de Parser}
\framesubtitle{Parser para atributos}
\begin{block}{}
\begin{hs}
pAttributes :: Parser [(String, String)]
pAttributes 
    = pListSep pSpaces pAttr
    where pAttr :: Parser (String, String)
          pAttr = (,) <$> pIdentifier 
                       <* pSymbol "=" 
                       <*> pQuotedString
\end{hs}
\end{block}
\end{frame}

\begin{frame}[fragile]
\frametitle{Definicion de Parser}
\framesubtitle{Parser para reconocer un tag}
\begin{block}{}
\begin{hs}
-- | Parser para reconocer un tag con inicio y fin
pTagged :: Parser NTree
pTagged 
    = do (tag,attrs) <- pTagInit
         elems       <- pList_ng pXML
         pTagEnd tag
         return $ buildTag elems tag attrs
\end{hs}
\end{block}
\end{frame}

\begin{frame}[fragile]
\frametitle{Definicion de Parser}
\framesubtitle{pTagInit y pTagEnd}
\begin{block}{}
\begin{hs}
-- | Parser para el tag de inicio
pTagInit :: Parser (String, [(String, String)])
pTagInit 
    = (,) <$ pSymbol "<" 
               <*> pIdentifier <*> pAttributes 
          <* pSym '>'

-- | Parser para el tag final
pTagEnd :: String -> Parser String
pTagEnd tag 
    = pSymbol "<" *> pSymbol "/" *> 
        (lexeme (pToken tag)) 
    <* pSym '>'
\end{hs}
\end{block}
\end{frame}

\begin{frame}[fragile]
\frametitle{Definicion de Parser}
\framesubtitle{Parser para reconocer un tag especial}
\begin{block}{}
\begin{hs}
-- | Parser para reconocer un tag especial (sin final)
pSpecialTag :: Parser NTree
pSpecialTag 
    = buildTag [] 
        <$ pSymbol "<" 
             <*> pIdentifier <*> pAttributes <*
           pSymbol "/" <* pSym '>'
\end{hs}
\end{block}
\end{frame}

\begin{frame}[fragile]
\frametitle{Definicion de Parser}
\framesubtitle{Parser para el texto}
\begin{block}{}
\begin{hs}
-- Parser para reconocer cualquier texto dentro de un tag
pText :: Parser NTree
pText = buildText 
         <$> pList1 (pSatisfy fcmp (Insertion "a" 'a' 5))
    where fcmp c = c /= '<'
\end{hs}
\end{block}
\end{frame}

\begin{frame}[fragile]
\frametitle{Definicion de Parser}
\framesubtitle{Funcion principal, pXML}

\begin{block}{}
\begin{hs}
-- | Parser que reconoce un tag normal/especial o texto
pXML :: Parser NTree
pXML = pText <|> pTagged <|> pSpecialTag
\end{hs}
\end{block}
\end{frame}

\begin{frame}[fragile]
\frametitle{Definicion de Parser}
\framesubtitle{Funcion principal, parser}

\begin{block}{}
\begin{hs}
-- | funcion principal para el parser xml
parser :: FilePath -> String -> NTree
parser fn inp = runParser fn p inp
    where p :: Parser NTree
          p = pSpaces *> pXML <* pSpaces
\end{hs}
\end{block}
\end{frame}


\frame{
  \vspace{2cm}
  {\huge Fin de la Presentación}

  \vspace{2cm}
  \begin{flushright}
    Carlos Gomez\\
    \structure{\footnotesize{carliros.g@gmail.com}}\\
  \end{flushright}
}

\end{document}
